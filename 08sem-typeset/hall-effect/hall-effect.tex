\documentclass[12pt,a4paper]{article}
\usepackage{../riemlabm}
\usepackage{fancyhdr}
\title{Hall Effect}
\lfoot{Regional Institute of Education Mysore}
\lhead{Hall Effect}
\chead{}
\rhead{Semester 8}
\cfoot{}
\rfoot{Physics}
\pagestyle{fancy}

\usepackage{listings}
\usepackage{color}

\definecolor{dkgreen}{rgb}{0,0.6,0}
\definecolor{gray}{rgb}{0.5,0.5,0.5}
\definecolor{mauve}{rgb}{0.58,0,0.82}

\lstset{frame=tb,
	language=Python,
	aboveskip=3mm,
	belowskip=3mm,
	showstringspaces=false,
	columns=flexible,
	basicstyle={\small\ttfamily},
	numbers=none,
	numberstyle=\tiny\color{gray},
	keywordstyle=\color{blue},
	commentstyle=\color{dkgreen},
	stringstyle=\color{mauve},
	breaklines=true,
	breakatwhitespace=true,
	tabsize=3
}

\begin{document}
		\maketitle
		
		\section{OBJECTIVE}
			\begin{itemize}
				\item 	To determine the hall voltage developed 
				across the sample material.
				
				\item 	To calculate the hall coefficient of the 
				sample material.
			\end{itemize}
		
		\section{Requirements}
		
		Two solenoids, constant power supply, four-probe digital 
		gauss meter, hall effect apparatus, digital multimeter and 
		hall probe
		
		\section{Introduction}
		
		When a magnetic field is applied perpendicular to a current 
		carrying conductor, a voltage is developed across the 
		specimen in a direction perpendicular to both the current and 
		the magnetic field. This phenomenon is called \emph{Hall 
		effect}. The voltage so developed is called the \emph{Hall 
		voltage}.
	
		
		
\end{document} 
