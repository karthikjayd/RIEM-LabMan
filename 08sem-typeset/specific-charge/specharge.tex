\documentclass[12pt,a4paper]{article}
\usepackage{../riemlabm}
\usepackage{fancyhdr}
\title{Specific charge of electron}
\lfoot{Regional Institute of Education Mysore}
\lhead{Specific charge of electron}
\chead{}
\rhead{Semester 8}
\cfoot{}
\rfoot{Physics}
\pagestyle{fancy}

\usepackage{listings}
\usepackage{color}

\definecolor{dkgreen}{rgb}{0,0.6,0}
\definecolor{gray}{rgb}{0.5,0.5,0.5}
\definecolor{mauve}{rgb}{0.58,0,0.82}

\lstset{frame=tb,
	language=Python,
	aboveskip=3mm,
	belowskip=3mm,
	showstringspaces=false,
	columns=flexible,
	basicstyle={\small\ttfamily},
	numbers=none,
	numberstyle=\tiny\color{gray},
	keywordstyle=\color{blue},
	commentstyle=\color{dkgreen},
	stringstyle=\color{mauve},
	breaklines=true,
	breakatwhitespace=true,
	tabsize=3
}

\begin{document}
		\maketitle
		
		\section{OBJECTIVE}
		
			To determine the specific charge $\left(\frac{e}{m}\right)$ of an electron.
			
		\section{REQUIREMENTS}
			Magnetometer, bar magnet, Cathode Ray Tube
			
		\section{INTRODUCTION}
			One of the major steps in explaining the composition of atoms was made by Joseph John Thomson by measuring the ratio of the electron's charge to its mass, known as the \textbf{specific charge} of an electron. The figure shows the various parts of the Thomson apparatus.
			
			Electrons are accelerated from the cathode to the anode, collimated by slits in the anodes, and then allowed to drift into a region of crossed (perpendicular) electric and magnetic fields. The simultaneously applied $\vec{E}$ and $\vec{B}$ fields are first adjusted to produce an
			undeflected beam. If the $\vec{B}$ field is then turned off, the $\vec{E}$ field alone
			produces a measurable beam deflection on the phosphorescent screen.
			From the size of the deflection and the measured values of $\vec{E}$ and $\vec{B}$, the
			charge-to-mass ratio, $\left(\frac{e}{m}\right)$ may be determined. The truly ingenious feature
			of this experiment is the manner in which Thomson measured $v_x$ , the
			horizontal velocity component of the beam. He did this by balancing the
			magnetic and electric forces. In effect, he created a \textit{velocity selector}, which
			could select out of the beam those particles having a velocity within a
			narrow range of values.
			
			
			
		\section{Observations}
		\begin{center}
			\begin{tabular}{|c|c|c|c|c|c|c|c|c|c|c|}
				\hline
				\rowcolor{b1!50}Distance&	\multicolumn{4}{c|}{Deflection}&	Mean&	$\tan\theta$&	$\tan^{2}\theta$&	$V$&	$y$&	$Vy$\\ \cline{2-5}
				\rowcolor{b1!50}$d$ (cm)&	$\theta_1$&	$\theta_2$&	$\theta_3$&	$\theta_4$&	$\theta$&	&	&	&	&	\\
				\rowcolor{b1!50}&	$(\degree)$&	$(\degree)$&	$(\degree)$&	$(\degree)$&	$(\degree)$&	&&(volts)&	(cm)&	V cm\\ \hline
				&&&&&&&&&& \\ \hline
				&&&&&&&&&& \\ \hline
				&&&&&&&&&& \\ \hline
				&&&&&&&&&& \\ \hline
				&&&&&&&&&& \\ \hline
				&&&&&&&&&& \\ \hline
				
			\end{tabular}
		\end{center}
	
	\section{Procedure}
	
	\section{Result}
	
		The dead time of the GM counter by the double source method is determined to be \rule{20ex}{0.2pt}ms.
\end{document} 
