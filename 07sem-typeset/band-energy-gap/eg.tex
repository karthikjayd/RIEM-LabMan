\documentclass[12pt,a4paper]{article}
\usepackage{../riemlabm}
\usepackage{fancyhdr}
\title{Band energy gap of a semiconductor}
\lfoot{Regional Institute of Education Mysore}
\lhead{Band energy gap of a semiconductor}
\chead{}
\rhead{Semester 7}
\cfoot{}
\rfoot{Physics}
\pagestyle{fancy}
\begin{document}
	\maketitle 
	
	\section{OBJECTIVE}
		
		\begin{enumerate}
			\item To study the variation of junction voltage with temperature.
			
			\item To determine the band energy gap of a semiconductor using a pn junction diode.
		\end{enumerate}
	
	\section{REQUIREMENTS}
		
		Electric oven, semiconductor diode (1N4007), thermometer, power supply, voltmeter.
	
	\section{THEORY}
		
		In a semiconductor, there exists an energy gap between its valence band and conduction band. A certain amount of energy has to be given to the electron so that it can go from the valence band to the conduction band. This energy required is the \textit{energy gap}, $E_G$ of the semiconductor between the two bands.
		
		When a pn junction is reverse biased, then the current is due to the minority carriers whose concentration is dependent on the energy gap $E_G$. The reverse current is a function of the temperature of the junction diode.
		
		$$V = \dfrac{E_G}{q} - \left[\ln\left(\dfrac{BT^{3}}{I_F}\right)\right]\dfrac{\eta k_{B} T}{q} $$
		
		$q$ is the electron charge, $k_B$ is
		Boltzmann’s constant, $\eta$ is a constant, normally in the range between 1 and 2,
		and $T$ is the absolute temperature of the p-n junction.
		
		For a definite value of $I$, the logarithmic term in the above equation can be considered constant, and the voltage $V$ is a linear function of $T$. From a family of $I-V$ characteristics obtained, one can find voltages related to a constant current I for different temperatures. According to the equation, the plot of the voltage $V$ versus the temperature $T$ should be a straight line. Extrapolation of this line to a zero temperature gives the bandgap energy $E_G$. 
		
		The experiment is carried out by measuring the voltage across a semiconductor diode by varying the temperature of the diode. The circuit diagram is given below.
		
	\section{\textcolor{b1}{PROCEDURE}}
		
		\begin{enumerate}
			
			\begin{figure}[!htb]
				\centering
				\includegraphics[scale=0.8]{Eg-circuit.pdf}
				\caption{Circuit Diagram}
			\end{figure}
			
			\item Connect the circuit as shown in figure 1. The circuit is almost identical to the I-V characteristics of diode experiment. Note that the diode should be connected in \textit{reverse bias}. The only addition is that the diode is kept in an electric oven for controlling the temperature of the diode.
			
			\item Insert the thermometer into the oven in which the diode is kept. Note down the room temperature and the corresponding voltage across the diode.
			
			\item Switch on the oven and let the temperature of the oven rise. With increase in temperature, note down the voltage across the diode for every $5\degree$ C rise in temperature. (Make sure that the current is constant at around $100\ \mu A$)
			
			\item As the temperature reaches about $80\degree$ C, turn off the oven and let the temperature fall. Continue noting the voltage readings for the same temperature values taken for heating and find the mean voltage.
			
			\item  Plot a graph of temperature (in kelvin) versus mean voltage (in volts).
			
		\end{enumerate}
	
	\section{\textcolor{b1}{OBSERVATIONS}}
		
		\begin{tabular}{|c|c|c|c|c|}
			\hline
			\multicolumn{5}{|c|}{Constant current $I= \rule{20ex}{0.2pt}$}\\ \hline
			\rowcolor{b1!25}Temperature&  Temperature&  \multicolumn{2}{c|}{Voltage $(V)$}&  Mean\\ \cline{3-4}  
			\rowcolor{b1!25}(in $\degree$ C)&  (in K)&  Heating&  Cooling&  voltage (V)\\ \hline
			
			$25$ &$25+273=298$ &&&\\ \hline
			$30$ &&&&\\ \hline
			$35$&&&&\\ \hline
			$40$&&&&\\ \hline
			$45$&&&&\\ \hline
			$50$&&&&\\ \hline
			$55$&&&&\\ \hline
			$60$&&&&\\ \hline
			$65$&&&&\\ \hline
			$70$&&&&\\ \hline
			$75$&&&&\\ \hline
		\end{tabular}
		\vspace{10pt}\\
		Constant current $I = \rule{10ex}{0.2pt}$
		
		\newpage
		
		\section{\textcolor{b1}{CALCULATIONS}}
			
			\begin{figure}[!htb]
				\centering
				\includegraphics[scale=0.7]{print.pdf}
				\caption{Sample graph plotted in gnuplot}
			\end{figure}
			
			The slope of the temperature ($T$) v/s junction voltage ($V$) plot gives a straight line with a negative slope as shown in figure 2.
			
			Using the straight line equation 
			
			$$y=mx+c$$
			
			The y-intercept $c$ of the straight line gives the value of $E_G$ in electron volts.
			
			
			\section{\textcolor{b1}{RESULT}}
			
			\begin{enumerate}
				\item	The junction voltage is found to \textit{decrease linearly} with temperature.
				
				\item	The band energy gap of the given semiconductor is determined to be \rule{10ex}{0.2pt}$eV$.
				
			\end{enumerate}
			
\end{document}