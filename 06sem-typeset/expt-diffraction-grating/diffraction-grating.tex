\documentclass[12pt,a4paper]{article}
\usepackage{../riemlabm}
\title{Diffraction grating}
\lfoot{Regional Institute of Education Mysore}
\lhead{Diffraction grating}
\chead{}
\rhead{Semester 6}
\cfoot{}
\rfoot{Physics}
\pagestyle{fancy} 

\begin{document}
	\maketitle
	\section{OBJECTIVE}
		To measure the wavelength of given light source by diffraction grating.
		
	\section{REQUIREMENTS}
		Diffraction grating, spectrometer, sodium discharge tube
	
	\section{PROCEDURE}
		\subsection{Initial Adjustments}
		\begin{enumerate}
			\item 	 First check leveling of the spectrometer base, prism table, collimator and telescope. If needed, level them using the adjustment screws and a spirit level.
			
			%\item	\textbf{The collimator:} The collimator is adjusted for parallel beam of light and the telescope for focusing the parallel beam. 
			
			\item 	\textbf{The telescope:} While looking through the telescope, slide the eyepiece in and out until the crosswire comes into sharp focus. Point the telescope at some distant object and view it through the telescope. Turn the focus knob of telescope until the image is sharp. The telescope is now focused for parallel light rays.\\
			\colorbox{yellow!25}{\emph{DO NOT change the focus of the telescope henceforth.}}
			
			\item 	Ensure the Hg lamp is fully illuminated and placed close to the slit of the collimator. Check that the slit is partially open.
			
			\item 	\textbf{The collimator:}  Align the telescope directly opposite the collimator and look through the telescope, to see a focused image of the slit. If necessary, adjust the slit width until the image of the slit as seen through the telescope is sharply focused on the crosswire. The collimator is then set to produce parallel light from the slit.
			
		\end{enumerate}
		
		\subsection{Setting the grating}
			\begin{enumerate}
				\item Focus	the telescope and collimator for parallel light. Turn telescope to P to directly view the image of the illuminated slit. Move the sodium source if necessary until the image is clear and bright; the source need not be very close to the slit.
				
				\begin{figure}[!htb]
					\centering
					\includegraphics[scale=0.6]{grating-1.pdf}
					\caption{}
				\end{figure}
				
				\item	Make the slit narrow and centre the image of the slit exactly on the crosswires. This may be done more accurately if the crosswires are set at $45\degree$ to the vertical. Read the vernier with the table locked.
				
				\item	 Turn the telescope to Q, so that the vernier reading is altered by exactly $90\degree$.
				
				\begin{figure}[!htb]
					\centering
					\includegraphics[scale=0.6]{grating-2.pdf}
					\caption{}
				\end{figure}
				
				\item	Place the diffraction grating in its holder on the table with its plane perpendicular to the line joining two of the screws L and M. Lines are ruled on the table as a guide. (The surface of G should not be touched; do not try to clean the surface.)
				
				\item 	Lock the telescope. Turn the table until image of slit after reflection from the surface of G appears. Adjust either of the screws L and M until the image in the surface of G appears in the centre of field of view.
				
				\item	Move the table until the image is centered on the crosswires. 
				Take the reading with the telescope locked.
				
				\item 	Turn the table so that the reading changes exactly by $45\degree$ and G is placed perpendicular to the light from C.
				
				\item	Turn the telescope to a position X to receive the first order diffracted image. If the grating lines are not perpendicular to the plane of rotation of the telescope the image may be displaced from the centre of the field of view. Adjust the third screw N to get the image at the centre.
			\end{enumerate}
		
		\subsection{Angle of diffraction}
			
			\begin{enumerate}
				\item	With the telescope at X, and the slit as narrow as possible, set the crosswires in turn to the two closely spaced components D$_{1}$ and D$_{2}$ lines. The D$_{1}$ line is (conventionally) that of longer wavelength, \textit{i.e.,} the more deviated line. 
				
				\item	Turn the telescope to Y, on the other side of the incident light and repeat the measurements.
				
				\begin{figure}[!htb]
					\centering
					\includegraphics[scale=0.6]{grating-3.pdf}
					\caption{}
				\end{figure}
				
				\item	Observe the second-order images (at larger angles) by turning the telescope. Record the readings for the settings on the D lines on either side of the normal for this order.
			\end{enumerate}			 
			
			\section{OBSERVATIONS}
				
				$$ \text{Least count} = \dfrac{0.5\degree}{30}=\dfrac{30'}{30}=1'$$
				
				\subsection{Initial Adjustments}
					
					\begin{itemize}
						\item	Reading with telescope at P, $x_{1}= \rule{10ex}{0.2pt}$
						\item	$x_{1}$ altered by $90\degree = \rule{10ex}{0.2pt}$
						
						\item	Reading with telescope at Q (the table adjusted), $x_{2}= \rule{10ex}{0.2pt}$
						\item	$x_{2}$ altered by $45\degree = \rule{10ex}{0.2pt}$
					\end{itemize}
			
				\subsection{Diffraction Angle}
				
				\begin{center}
					\begin{tabular}{|c|>{\centering\arraybackslash}p{30pt}|>{\centering\arraybackslash}p{30pt}|>{\centering\arraybackslash}p{30pt}|>{\centering\arraybackslash}p{30pt}|>{\centering\arraybackslash}p{30pt}|>{\centering\arraybackslash}p{30pt}|>{\centering\arraybackslash}p{30pt}|>{\centering\arraybackslash}p{30pt}|}
					\hline
					\rowcolor{b1!25}&\multicolumn{4}{|c|}{First Order}& \multicolumn{4}{c|}{Second Order}\\ \cline{2-9}
					
					\rowcolor{b1!25}& \multicolumn{2}{c|}{Vernier Angles}& && \multicolumn{2}{c|}{Vernier Angles}&&\\ \cline{2-3} \cline{6-7}
					
					\rowcolor{b1!25}& At X& At Y& $2\theta$& $\theta$& At X& At Y& $2\theta$& $\theta$ \\ 
					
					\rowcolor{b1!25}& left& right& && left& right& & \\ \hline
					
					Line D$_{1}$& & & & & & & & \\ 
					&&&&&&&&\\ \hline
					Line D$_{2}$& & & & & & & & \\
					&&&&&&&&\\ \hline
					
				\end{tabular}
				\end{center}
			
			\section{CALCULATIONS}
				
				Grating spacing $= \dfrac{1}{\text{no. of lines per cm}}=\rule{10ex}{0.2pt}$ m
				
				\subsection{At D$_{1}$}
					For first order,
						$$\lambda_{1}=d\sin\theta=\rule{10ex}{0.2pt}\text{ \AA}$$
						
					For second order,
						$$2\lambda_{1}=d\sin\theta'$$
						$$\lambda_{1}=\rule{10ex}{0.2pt}\text{ \AA}$$
				
				\subsection{At D$_{2}$}
					For first order,
					$$\lambda_{2}=d\sin\theta=\rule{10ex}{0.2pt}\text{ \AA}$$
					
					For second order,
					$$2\lambda_{2}=d\sin\theta'$$
					$$\lambda_{2}=\rule{10ex}{0.2pt}\text{ \AA}$$
			
			\section{RESULT}
			The wavelength of the two sodium lines were found to be \\
			$$\lambda_{1} (D_{1}) = \rule{20ex}{0.2pt}\text{ \AA}$$
			$$\lambda_{2} (D_{2}) = \rule{20ex}{0.2pt}\text{ \AA}$$
\end{document}